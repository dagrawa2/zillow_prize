\documentclass{beamer}
\usepackage{amsmath, amsfonts, amssymb}
\usepackage{graphicx}
%\usepackage{pgfplots, pgfplotstable}
\mode<presentation>
{\usetheme{boxes}}
\setbeamertemplate{frametitle}[default][center] 

\usepackage{lmodern, exscale}

\usepackage{xcolor}

%\newcommand{\RR}{\mathbb{R}}
%\newcommand{\eps}{\epsilon}
%\newcommand{\lmat}{\begin{bmatrix}}
%\newcommand{\rmat}{\end{bmatrix}}
%\newcommand{\argmin}{\operatorname{argmin}}
%\newcommand{\fnn}{f_{\mbox{NN}}}
%\newcommand{\tilw}{\tilde{w}}
%\newcommand{\tilW}{\tilde{W}}

\title{Zillow prize: Predicting the error in Zestimates}
\author{Devanshu Agrawal}
\date{September 17, 2018}

\begin{document}

\begin{frame}
\titlepage
\end{frame}

\begin{frame}
\frametitle{Introduction}
\begin{itemize}
\item Zillow: Online real-estate database company.
\item Zestimate: Estimated home value based on machine learning and statistical models.
\item Zillow Prize: 2017 Kaggle competition to predict the log-error in Zestimates:
\[ \mbox{log-error} = \log(\mbox{Zestimate}) - \log(\mbox{sale price}). \]
\end{itemize}
\end{frame}

\begin{frame}
\frametitle{Data Files}
\begin{itemize}
\item Data of homes from three counties in southern California.
\begin{itemize}
\item properties\_2016.csv
\item transactions\_2016.csv
\item properties\_2017.csv
\item transactions\_2017.csv
\end{itemize}
\item Will focus the discussion on 2016 data.
\end{itemize}
\end{frame}

\begin{frame}
\frametitle{Properties Data}
\begin{itemize}
\item Contains 2985217 properties with 57 features not including ``parcelid''.
\end{itemize}
\begin{center}
\footnotesize
\begin{tabular}{|c|c|c|c|c|c|c|} \hline
parcelid & airconditioning & architectural & basement & bathroom & bedroom & $\cdots$ \\ \hline
\quad & typeid & styletypeid & sqft & cnt & cnt & \quad \\ \hline
$\vdots$ & $\vdots$ & $\vdots$ & $\vdots$ & $\vdots$ & $\vdots$ & $\ddots$ \\
13037353 & 1.0 & nan & nan & 3.0 & 4.0 & $\cdots$ \\
13037414 & 1.0 & nan & nan & 2.0 & 3.0 & $\cdots$ \\
13037460 & 1.0 & nan & nan & 4.0 & 5.0 & $\cdots$ \\
13037644 & nan & nan & nan & 2.0 & 2.0 & $\cdots$ \\
13037744 & nan & nan & nan & 3.0 & 4.0 & $\cdots$ \\
$\vdots$ & $\vdots$ & $\vdots$ & $\vdots$ & $\vdots$ & $\vdots$ & $\ddots$ \\
\hline
\end{tabular}
\end{center}
\end{frame}

\begin{frame}
\frametitle{Transactions Data}
\begin{itemize}
\item Contains 90275 transactions but only 90150 unique properties.
\end{itemize}
\begin{center}
\begin{tabular}{|c|c|c|} \hline
parcelid & logerror & transactiondate \\ \hline
$\vdots$ & $\vdots$ & $\vdots$ \\
14660592 & -0.0151 & 2016-01-07 \\
14129273 & 0.0334 & 2016-01-07 \\
11078035 & 0.002 & 2016-01-07 \\
12058667 & 0.004 & 2016-01-07 \\
12660997 & 0.0392 & 2016-01-07 \\
$\vdots$ & $\vdots$ & $\vdots$ \\
\hline
\end{tabular}
\end{center}
\end{frame}

\begin{frame}
\frametitle{The Problem}
\begin{itemize}
\item Build a model that predicts the log-error in Zestimate for all 2985217 properties for months October, November, and December in years 2016 and 2017.
\item Zillow has withheld the log-errors for some transactions in October, November, and December; predictions for these transactions are used to score participants.
\begin{itemize}
\item 2016 predictions determine public leaderboard (LB) and 2017 predictions determine private LB.
\end{itemize}
\end{itemize}
\end{frame}

\begin{frame}
\frametitle{Outline of the Talk}
\begin{itemize}
\item Data imputation and exploration.
\item Dimensionality reduction.
\item Model descriptions.
\item Results.
\item Discussion.
\end{itemize}
\end{frame}

\begin{frame}
\frametitle{Missing Values}
\begin{itemize}
\item Percentage of values missing for each feature.
\end{itemize}
\begin{center}
\footnotesize
\begin{tabular}{|c|c|c|} \hline
\quad & Feature & Fraction missing \\ \hline
1 & storytypeid & 0.9995 \\
2 & basementsqft & 0.9995 \\
3 & yardbuildingsqft26 & 0.9991 \\
%4 & fireplaceflag & 0.9983 \\
%5 & architecturalstyletypeid & 0.998 \\
%6 & typeconstructiontypeid & 0.9977 \\
%7 & finishedsquarefeet13 & 0.9974 \\
%8 & buildingclasstypeid & 0.9957 \\
%9 & decktypeid & 0.9943 \\
%10 & finishedsquarefeet6 & 0.9926 \\
$\vdots$ & $\vdots$ & $\vdots$ \\
%26 & airconditioningtypeid & 0.7282 \\
%27 & garagecarcnt & 0.7041 \\
28 & garagetotalsqft & 0.7041 \\
29 & regionidneighborhood & 0.6126 \\
30 & heatingorsystemtypeid & 0.3949 \\
$\vdots$ & $\vdots$ & $\vdots$ \\
%53 & fips & 0.0038 \\
%54 & regionidcounty & 0.0038 \\
55 & latitude & 0.0038 \\
56 & longitude & 0.0038 \\
57 & rawcensustractandblock & 0.0038 \\
\hline
\end{tabular}
\end{center}
\end{frame}

\begin{frame}
\frametitle{Imputation Strategy}
\begin{itemize}
\item Group features into rough categories: Bathroom features, finished area features, tax-related features, etc.
\item Go through each group and see if some features can be used to help impute other features.
\item Two kinds of missing values:
\begin{itemize}
\item Missing value indicates some default null value.
\item Value is missing for unknown reason.
\end{itemize}
\end{itemize}
\end{frame}

\begin{frame}
\frametitle{Location Features}
\begin{itemize}
\item Includes ``regionidcity'', ``regionidcounty'', ``regionidzip'', ``regionidneighborhood'', ``latitude'', and ``longitude''.
\item If ``latitude'' or ``longitude'' is missing, then all other location features are missing.
\item Imputed ``latitude'' and ``longitude'' with their modes.
\begin{itemize}
\item Mean or median would not have guaranteed coordinates in residential area.
\end{itemize}
\item Retreived missing zipcodes using coordinates and the ``uszipcode'' python module.
\end{itemize}
\end{frame}

\begin{frame}
\frametitle{Location Features (cont.)}
\begin{itemize}
\item Properties with the same zipcode should be in the same county.
\end{itemize}
\begin{center}
\begin{tabular}{|c|c|c|} \hline
Property & Zipcode & County \\ \hline
A & 96450 & 3101 \\
B & 96450 & nan \\
\hline
\end{tabular}
\end{center}
\end{frame}

\begin{frame}
\frametitle{Location Features (cont.)}
\begin{itemize}
\item Imputed ``regionidcity'' with similar strategy as for ``regionidcounty''.
\item All properties with zipcode 97331 were missing ``regionidcity''.
\begin{itemize}
\item Imputed final missing value of ``regionidcity'' with florred mean.
\item Confirmed that imputed city index was unique.
\end{itemize}
\item Dropped ``regionidneighborhood'' since missing over 60\% of values and no strategy for imputing.
\end{itemize}
\end{frame}

\begin{frame}
\frametitle{Bathroom Features}
\begin{itemize}
\item Includes ``bathroomcnt'', ``calculatedbathnbr'', ``fullbathcnt'', and ``threequarterbathnbr''.
\item Identified multicolinearity:
\[ (\mbox{calculatedbathnbr}) = (\mbox{fullbathcnt}) + 0.5(\mbox{threequarterbathnbr}). \]
\begin{itemize}
\item Three-qquarter bathrooms are actually half bathrooms.
\item Used relationship to help impute missing values.
\item Dropped ``fullbathcnt''.
\end{itemize}
\item Features ``bathroomcnt'' and ``calculatedbathnbr'' are equal whenever both values are known.
\begin{itemize}
\item Dropped ``calculatedbathnbr'' since  it had more missing values.
\end{itemize}
\end{itemize}
\end{frame}


\begin{frame}
\frametitle{Fireplace Features}
\begin{itemize}
\item Includes ``fireplaceflag'' and ``fireplacecnt''.
\item In ``fireplaceflag'', convert ``true'' and ``false'' to $1$ and $0$ respectively.
\item Impute ``fireplacecnt'' to $0$ if ``fireplaceflag'' is $0$ and to the median otherwise.
\item Impute ``fireplaceflag'' to $1$ if ``fireplacecnt'' is positive and otherwise to $0$.
\end{itemize}
\end{frame}

\begin{frame}
\frametitle{Final Training Set}
\begin{itemize}
\item Dropped a total of 19 features due to one of the following:
\begin{itemize}
\item Too many missing values or difficult to impute.
\item Redundancy; another feature is identical.
\item Multicolinearity; linear combination of other features.
\item Provides little useful information.
\end{itemize}
\item Added ``month'' feature.
\item Final training set has 39 features.
\end{itemize}
\end{frame}

%\item Formed training set (X\_train\_2016, y\_train\_2016):
%\begin{itemize}
%\item X\_train\_2016: Augment transactions\_2016.csv with features from properties\_2016.csv, replace transaction date with month, and drop log-error.
%\item y\_train\_2016: extract log-error from transactions\_2016.csv.
%\end{itemize}
%\end{itemize}
%\end{frame}

%%\begin{frame}
%\frametitle{Final Training Set (cont.)}
%\begin{itemize}
%\item All properties in X\_train\_2016 have same value for ``taxassessmentyear''.
%\item Dropped ``taxassessmentyear'' from X\_train\_2016.
%\item (X\_train\_2016, y\_train\_2016) has 90275 training samples of dimension 39.
%\item Prepared (X\_train\_2017, y\_train\_2017) similarly.
%\end{itemize}
%\end{frame}

\begin{frame}
\frametitle{Correlations with the Target}
\begin{itemize}
\item Poor correlation of each feature with log-error.
\end{itemize}
\begin{center}
\includegraphics[width=3in]{../results/explore/2016/plots/corrs_with_y.png}
\end{center}
\end{frame}

\begin{frame}
\frametitle{A Cross-Section of Data}
\begin{itemize}
\item Noise likely due to variation of log-error with the other 38 features.
\item Correlation gives poor measure of importance.
\end{itemize}
\begin{center}
\includegraphics[width=3in]{../results/explore/2016/plots/cs-calculatedfinishedsquarefeet.png}
\end{center}
\end{frame}

\begin{frame}
\frametitle{Alternative Approach to Correlations}
\begin{itemize}
\item Compute correlation between feature $A$ and target while holding all other features fixed (or restricted to small bin of values).
\item Average resulting correlations over all slices of the other features.
\end{itemize}
\begin{center}
\includegraphics[width=3in]{plots/alt.png}
\end{center}
\end{frame}

\begin{frame}
\frametitle{Correlations between Features}
\begin{center}
\footnotesize
\begin{tabular}{|c|c|c|c|} \hline
\quad & Feature 1 & Feature 2 & Pearson \\ \hline
1 & poolcnt & poolsizesum & 0.9966 \\
2 & poolcnt & pooltypeid7 & 0.9579 \\
3 & taxvaluedollarcnt & landtaxvaluedollarcnt & 0.9553 \\
4 & poolsizesum & pooltypeid7 & 0.9549 \\
5 & calculatedfinishedsquarefeet & finishedsquarefeet50 & 0.9405 \\
6 & fireplacecnt & fireplaceflag & 0.919 \\
7 & garagecarcnt & garagetotalsqft & 0.8398 \\
%8 & structuretaxvaluedollarcnt & taxvaluedollarcnt & 0.8242 \\
$\vdots$ & $\vdots$ & $\vdots$ & $\vdots$ \\
737 & fips & regionidcounty & -0.6091 \\
738 & regionidcounty & roomcnt & -0.6376 \\
739 & garagetotalsqft & regionidcounty & -0.6695 \\
740 & heatingorsystemtypeid & regionidcounty & -0.7861 \\
741 & garagecarcnt & regionidcounty & -0.8538 \\
\hline
\end{tabular}
\end{center}
\end{frame}

\begin{frame}
\frametitle{Dimensionality Reduction}
\begin{itemize}
\item Prepared reduced data set in addition to original data set.
\item Principal components analysis (PCA):
\begin{itemize}
\item Selected 25 principal components; explains 95\% of variance.
\end{itemize}
\end{itemize}
\begin{center}
\includegraphics[width=3in]{../results/explore/2016/plots/evr.png}
\end{center}
\end{frame}

\begin{frame}
\frametitle{Regression Models}
\begin{itemize}
\item Algorithms:
\begin{itemize}
\item Random forest (RF).
\item Gradient-boosted decision tree (GBDT).
\end{itemize}
\item Libraries:
\begin{itemize}
\item LightGBM.
\item Sci-KitLearn.
\item Hyperopt
\end{itemize}
\end{itemize}
\end{frame}

\begin{frame}
\frametitle{Random Forests}
\begin{itemize}
\item Want model that minimizes mean squared error:
\[ \mbox{MSE} = \mbox{bias}^2 + \mbox{variance}. \]
\item Bias, variance, and MSE are expectations over space of training sets (of given size).
\item Many training sets are better than one.
\item Can simulate many training sets through bootstrapped aggregation (bagging).
\end{itemize}
\end{frame}

\begin{frame}
\frametitle{Random Forests (cont.)}
\begin{itemize}
\item RF: Ensemble of decision trees.
\item Each base learner trained on random boostrapped sample of data (bagging) and on random fraction of features.
\item Prediction is average of all individual predictions.
\item Ensemble reduces variance by giving better approximation to MSE.
\end{itemize}
\end{frame}

\begin{frame}
\frametitle{Gradient-Boosted Decision Trees}
\begin{itemize}
\item Consider data $(X, y)$, loss $L(y, f)$, and decision tree $f_m$ fitted to data.
\item Fit decision tree $h_m$ to gradient data $(X, \nabla_f L(y, f_m))$.
\item Perform one step of gradient descent with line search and shrinkage to obtain boosted decision tree:
\[ f_{m+1} = f_m - \alpha\gamma\cdot h_m. \]
\item Separate line search parameter $\gamma_i$ for each leaf of $h_m$.
%\item Component $\gamma_i$ of $\gamma$ is the fitted line search parameter for the $i$th leaf of $h_m$.
\item $\alpha$ is the shrinkage hyperparameter (learning rate).
\end{itemize}
\end{frame}

\begin{frame}
\frametitle{Gradient-Boosted Decision Trees (cont.)}
\begin{itemize}
\item Boosting is iterative.
\item Each base learner is fit to a random subsample of training data and on a random fraction of features.
\begin{itemize}
\item Analogous to stochastic gradient descent with dropout in neural networks.
\item Helps to combat overfitting.
\item Speeds up training.
\end{itemize}
\end{itemize}
\end{frame}

\begin{frame}
\frametitle{LightGBM}
\begin{itemize}
\item Newer alternative to XGBoost for ensemble and gradient-boosted models.
\item Faster and more efficient.
\begin{itemize}
\item During subsampling, training samples resulting in large gradients are kept.
\item Trees are grown leaf-wise instead of level-wise.
\begin{itemize}
\item Leads to more complex trees that reduce loss more efficiently.
\item Increases chances of overfitting.
\end{itemize}
\end{itemize}
\end{itemize}
\end{frame}

\begin{frame}
\frametitle{Key Hyperparameters}
\begin{itemize}
\item Hyperparameters common to RFs and GBDTs:
\begin{itemize}
\item n\_estimators.
\item num\_leaves.
\item feature\_fraction.
\item bagging\_fraction.
\end{itemize}
\item Additional hyperparameters for GBDTs:
\begin{itemize}
\item learning\_rate.
\item lambda\_l1.
\item lambda\_l2.
\end{itemize}
\end{itemize}
\end{frame}

\begin{frame}
\frametitle{Hyperparameter Optimization}
\begin{itemize}
\item Used 3-fold cross validation.
\item Scored model with mean absolute error averaged over 3 test folds.
\item Used the Tree-Structure Parzen Estimation (TPE) algorithm.
\begin{itemize}
\item Declare prior distribution over hyperspace.
\item Sample hyperparameters and evaluate model with each sample.
\item Use Bayes' Rule to obtain posterior distribution.
\item Iterate.
\end{itemize}
\item Optimized ``n\_estimators'' for GBDT with early stopping.
\end{itemize}
\end{frame}

\begin{frame}
\frametitle{Results}
\begin{center}
\footnotesize
\begin{tabular}{|c|c|c|c|} \hline
Model & Public LB & Private LB & Notes \\ \hline
\textcolor{blue}{GBDT0} & \textcolor{blue}{0.0651179} & \textcolor{blue}{0.0759204} & \textcolor{blue}{baseline, very different} \\
GBDT1 & 0.0647788 & 0.0763565 & separate HO on 2016, 2017 \\
GBDT2 & 0.0647788 & 0.0762839 & transfer 2016 hyperparams to 2017 \\
GBDT3 & 0.0648313 & 0.0763703 & PCA \\
GBDT4 & 0.0647504 & 0.0761217 & hand-picked hyperparams \\
GBDT5 & 0.0647539 & 0.0761002 & bagging, L\_1, L\_2 \\
RF1 & 0.0647544 & 0.0760869 & very slow \\
\hline
\end{tabular}
\end{center}
\end{frame}

\begin{frame}
\frametitle{Remarks on Results}
\begin{itemize}
\item Baseline model achieves best private LB.
\begin{itemize}
\item Was meant only for testing code.
\item Data preprocessed differently; fewer features dropped.
\item 2016 predictions were copied to 2017.
\end{itemize}
\item PCA resulted in worse performance.
\item Hypothesis: Reduced multicolinearity caused the model to underfit.
\end{itemize}
\end{frame}

\begin{frame}
\frametitle{Decision Tree Toy Example}
\begin{itemize}
\item Consider binary classification problem.
\item Decision boundary is $x+y=0$ with $x+y>0$ class $1$ and $x+y<0$ class $0$.
\end{itemize}
\begin{center}
\includegraphics[width=3in]{plots/toy-1.png}
\end{center}
\end{frame}

\begin{frame}
\frametitle{Decision Tree Toy Example (cont.)}
\begin{itemize}
\item Decision trees partition feature space with hyperplanes perpendicular to the coordinate axes.
\item A very deep decision tree can estimate $x+y=0$ with a ``staircase''.
\end{itemize}
\begin{center}
\includegraphics[width=3in]{plots/toy-2.png}
\end{center}
\end{frame}

\begin{frame}
\frametitle{Decision Tree Toy Example (cont.)}
\begin{itemize}
\item Engineer a new feature $z = x+y$.
\item Then $z=0$ is an optimal decision bounary and is perpendicular to the $z$-axis.
\item Thus $z=0$ can be learned by a decision stump.
\item Introduction of a colinear feature reduced tree complexity from infinite to minimal.
\item Discarding features or performing PCA to reduce multicolinearity might cause tree-based models to underfit.
\end{itemize}
\end{frame}

\begin{frame}
\frametitle{Feature Importances}
\begin{itemize}
\item Feature engineering should focus on most important features.
\end{itemize}
\begin{center}
\footnotesize
\begin{tabular}{|c|c|c|c|c|} \hline
\quad & GBDT5 & RF1 & \\
1 & taxpercentage & taxpercentage \\
2 & yearbuilt & calculatedfinishedsquarefeet \\
3 & calculatedfinishedsquarefeet & finishedsquarefeet50 \\
4 & structuretaxvaluedollarcnt & latitude \\
5 & regionidzip & structuretaxvaluedollarcnt \\
6 & taxvaluedollarcnt & landtaxvaluedollarcnt \\
7 & finishedsquarefeet50 & yearbuilt \\
8 & landtaxvaluedollarcnt & taxvaluedollarcnt \\
%9 & lotsizesquarefeet & longitude \\
%10 & latitude & regionidzip \\
$\vdots$ & $\vdots$ & $\vdots$ \\
35 & fireplacecnt & yardbuildingsqft17 \\
36 & yardbuildingsqft17 & yardbuildingsqft26 \\
37 & decktypeid & decktypeid \\
38 & yardbuildingsqft26 & pooltypeid2 \\
39 & basementsqft & basementsqft \\
\hline
\end{tabular}
\end{center}
\end{frame}

\begin{frame}
\frametitle{Future Improvements}
\begin{itemize}
\item For tree-based models, discard features only if redundant, error-prone, or difficult to impute.
\item Perform feature engineering.
\item Perform more extensive HO.
\item Specify categorical features.
\item Consider the geometry of the data.
\end{itemize}
\end{frame}

\end{document}